\section{Conclusions}

After applying three methods, $k$-NN, SVM, random forest, to Seattle climate dataset, SVM and random forest have similar results in terms of overall error rate and are better than $k$-NN. SVM has a kind of balance between type \uppercase\expandafter{\romannumeral1} and type \uppercase\expandafter{\romannumeral2} error, while random forest has larger type \uppercase\expandafter{\romannumeral2} error and lower type \uppercase\expandafter{\romannumeral1} error. So random forest may make more mistakes in predicting days with precipitation. In reality people may care more and change their action, like bring an umbrella, if they know it will rain tomorrow. So in that case SVM would be better than random forest. Also the results of applying the model to New York data suggest we can’t use the models directly on another city without modification. Overall, the classification methods are reliable if we have enough history data and just want to make prediction of precipitation. 