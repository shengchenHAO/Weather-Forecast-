\section{Introduction}

Weather forecast is important for people to make their schedule for the next day. Among those technical terms in the forecasting, temperature and precipitation are the most important for the public. Nowadays weather forecasts, especially the precipitation forecast is conducted with the help of satellite images of clouds. This requires knowledge in meteorology. So we wonder if we can predict whether it will rain tomorrow based on the basic weather record (like temperature, precipitation, wind), with the classification method we learned in class.

We choose Seattle for our study as it usually has half of time in a year raining, so the data are balanced. In order to make better forecasting, feature engineering is needed. Also the data comes in form of daily report by each station. Combining the weather report each day from different stations is also a challenge for our study.  The location of weather stations and relationship between variables are discussed in the section of data exploration. And we are going to use Random Forest, $k$-NN and SVM for classification. At last part we are going to use the models trained with Seattle data to predict precipitation in another city to see whether the models are generally usable.