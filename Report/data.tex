\subsection{Dataset in Use}

The dataset used in our study is obtained from \href{https://www.ncdc.noaa.gov/cdo-web/}{NOAA} (National Oceanic and Atmospheric Administration). It is a part of NOAA's GHCN(Global Historical Climatology Network)-Daily database\cite{noaa}, and is concerned with daily climate data of Seattle from January \engordnumber{1}, 2012 to January \engordnumber{1}, 2018. The records are colloected from up to 190 stations accross Seattle area, with different sets of attributes. The main attributes are listed in Table \ref{tbld}.

For practical use, we sample $70\%$ of the data as the training set, and the rest as the test set.

The models are also applied to daily climate data of New York in 2017, which come from the same database.

\begin{table}[h]
\setlength{\belowcaptionskip}{5pt}
\caption{Attributes of the Dataset}
\label{tbld}
\centering
\renewcommand\arraystretch{1.5}
\begin{tabular}{|c|l|}
\hline
\textbf{Attribute} & \textbf{Explanation (unit)}\\
\hline
PRCP & Precipitation (inches)\\
\hline
TMAX & Maximum temperature ($^\circ$F)\\
\hline
TMIN & Minimum temperature ($^\circ$F)\\
\hline
TOBS & Temperature at the time of observation ($^\circ$F)\\
\hline
WDF2 & Direction of fastest 2-minute wind ($^\circ$)\\
\hline
WDF5 & Direction of fastest 5-second wind ($^\circ$)\\
\hline
WSF2 & Fastest 2-minute wind speed (miles/h)\\
\hline
WSF5 & Fastest 5-second wind speed (miles/h)\\
\hline
AWND & Average daily wind speed (miles/h)\\
\hline
\end{tabular}
\end{table}